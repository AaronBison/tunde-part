\section{Webes keretrendszerek}
A keretrendszerek \cite{frameworks} napjainkban felkapott rendszerek lettek a felhasználók között. Ezek a rendszerek sokoldalúak, robusztusak és hatékonyak. A különböző alkalmazások fejlesztéséhez az ilyen jellegű rendszerek segítségével csak a magasabb szintű funkcionalitások elvégzésére kell koncentrálni. Erre az a magyarázat, hogy a keretrendszer gondoskodik az alacsonyabb funkcionalitásokról, amelyek már rengeteg tesztelt kódot tartalmaznak, így mi ezeket a funkcionalitásokat nem kell külön megírjuk és leteszteljük, hogy helyes-e a megírt kód. Számos előnyt lehet felsorolni, hogy miért jó ha használjuk ezeket a rendszereket:
\begin{itemize}
	\item Elősegíti a tervezési minták megfelelő kialakítását.
	\item Biztonságosabb kódolás.
	\item A redundáns kód elkerülése.
	\item Következetes kódfejlesztés kevesebb hibával.
	\item Megkönnyíti a kód tesztelését és a hibakeresést is.
	\item Az alkalmazás fejlesztéséhez szükséges idő lecsökken.
\end{itemize}

Az \ref{fig:fornt_frameworks} ábrán látható egy rangsorolás a Github és StackOverflow által készített pontozások alapján a Front-end-et megvalósító keretrendszereknek. A Front-end foglalkozik a kliens felületek megjelenítésével. Az ábrán látható hogy az első helyen a React van összesített 98 ponttal. A második helyen áll az ASP.NET MVC 95 ponttal viszont itt észlelhető az is, hogy a Githubról nem jött pontozás erről a keretrendszerről. Az ábrán látható további keretrendszerek úgyan annyi összpontozást kaptak. Az ábra alapján a legjobb döntés egy webes keretrendszer választásra a React lenne visszont a többi keretrendszer sincs sokkal lemaradva az első helyezettől.
\begin{figure}[H]
	\centering
	\includegraphics[width=1\linewidth]{figures/images/vueang.png}
	\caption[Webes keretrendszerek Github és Stack Overflow pontozások alapján]{\textit{Webes keretrendszerek Github és Stack Overflow pontozások alapján}\footnotemark}
	\label{fig:fornt_frameworks}
\end{figure}
\footnotetext{http://hotframeworks.com/}

2021-ben a 10 leghasználatosabb Back-end keretrendszer a Spring - Spring Boot, NodeJS, Laravel, Django, Flask, Ruby, Play, Asp.Net, CakePHP és Symphony \cite{backendframework1}\cite{backendframework2}. A Back-end segít a Front-end-nek egy dinamikus alkalmazás elkészítésében. Ez azt jelenti, hogy a Back-end biztosítja az adatokat a Front-end-nek.

A következőkben olvashatunk részletesebben néhány keretrendszerről mint például: az Angular, Vue.js, Spring Boot és a NodeJs.

\subsection{Angular}
A Google munkatársai 2008-ban fejlesztettek a JavaScript alapú Angular keretrendszert \cite{wohlgethan2018supportingweb}. Abban az időben a webhelyek zöme többoldalas alkalmazás megközelítésén alapult. A több oldalas alkalmazások lassúnak bízonyultak így bevezették az egy oldalas alkalmazásokat. Az egy oldalas alkalmazások abban jobbak a több oldalas alkalmazásoknál, hogy weboldal betöltése folyamán nem kell mindig mindent újra betölteni, hanem csak azok a részek töltődnek be ahol változások fognak megjelenni. Az Angular volt az egy oldalas alkalmazások első kerete. Az egyik fő előnye, hogy a felhasználók egyszerű struktúrával kell dolgozzanak. Megtanulják az Angular sajátos felépítését ezáltal gyorsan és optimálisabban tudnak benne fejleszteni. Az sem elhanyagolható, hogy a fejlesztők egy részletes és egyértelmű dokumentációval szolgáltak a felhasználóknak.

\subsection{Vue.js}
A Vue.js (röviden: Vue) \cite{wohlgethan2018supportingweb} tekinthető az egyik legújabb keretrendszernek. Hasonlít az Angularhoz, mindkétt keretrendszer TypeScript típusú. Használható kisebb, egyszerűbb projekteknél. Mindig egy oldalas alkalmazásokat lehet benne elkészíteni ami a felhasználói élményt segíti elő. Fő érdeme a skálázhatóság és különlegessége, hogy egy nyílt forráskódú közösség fejlesztette ki, nem pedig egy nagyobb vállalat, így a problémák megoldására rengeteg segítséget lehet kapni. Komponens alapú keretrendszer amely azt jelenti, hogy komponenseket különböztetünk és jelenítünk meg. Egy komponensen belül tudunk írni HTML, CSS és Script elemeket is. A megjelenítés egy oldalon történik ezért szükséges használni a útválasztást (rout).
	
\subsection{Angular vs Vue.js}
Mindkét keretrendszernek megvannak az előnyei \ref{tab:table1} táblázat és a hátrányai is \ref{tab:table2} táblázat. 
\begin{table}[H]
	\begin{footnotesize}
		\begin{center}
			\caption[Előnyök Angular és Vue.js között]{\textit{Előnyök Angular és Vue.js között \cite{vuevsang}}}
			\label{tab:table1}
			\begin{tabular}{p{0.5cm}|p{6cm}|p{6cm}}
				\textbf & \textbf{Angular} & \textbf{Vue.js}\\
				\hline
				1 & TypeScript használata & TypeScript használata, részletes dokumentáció\\
				\hline
				2 & Részletes dokumentációval rendelkezik & Egy oldalas alkalmazások készítése \\
				\hline
				3 & Gyorsítja a fejlesztést & Könnyű integráció a meglévő struktúrákba \\
				\hline
				4 & & Kihasználja virtuális DOM előnyeit \\
				\hline
				5 & & Sebessége és rugalmassága optimális \\
			\end{tabular}
		\end{center}
	\end{footnotesize}
\end{table}
\begin{table}[H]
	\begin{footnotesize}
		\begin{center}
			\caption[Hátrányok Angular és Vue.js között]{\textit{Hátrányok Angular és Vue.js között \cite{vuevsang}}}
			\label{tab:table2}
			\begin{tabular}{p{0.5cm}|p{6cm}|p{6cm}} 
				\textbf & \textbf{Angular} & \textbf{Vue.js}\\
				\hline
				1 & Számos különféle struktúrát kínál, nehezíti a tanulást & Kevesebb erőforrást kínál\\
				\hline
				2 & Lassabb teljesítmény mert működik a reális DOM &  \\
			\end{tabular}
		\end{center}
	\end{footnotesize}
\end{table}

\subsection{NodeJs}
A NodeJs \cite{js2016node} egy olyan szoftver platform, amely a Chrome V8 JavaScript futási idején épül fel. Fontos tulajdonsága, hogy skálázható ezért is sokan használják. Eseményvezérelt, nem blokkoló I/O modellt használ, amely könnyűvé és hatékonnyá teszi a valós idejű alkalmazások megvalósítását, amelyek elosztott rendszeren futnak át. Tulajdonképpen közvetlenül a natív gépi kódra fordítja le a JavaScript-et, amelyenek hatására az adatformátum egy lépése megszűnik, ezzel növelve az alkalmazás sebességét. Legtöbb esetben a NodeJs használatakor adatbázisnak NoSQL adatbázisokat választanak.

NodeJs a következő különlegességeket tartalmazza \cite{nodejsspring}:
\begin{itemize}
	\item A NodeJS alkalmazások fejlesztésének elindítása könnyű.
	\item Agilis fejlesztési módszertant követi.
	\item Alkalmas a skálázható alkalmazásfejlesztési szolgáltatásokra.
	\item Nagy projekteknél gyorsabban működik mint a Java.
	\item Hatalmas könyvtárakkal rendelkezik, amelyek segítik a fejlesztőket.
\end{itemize}
	
\subsection{Spring Boot}
A Spring keretrendszer \cite{spring} egy programozási és konfigurációs modellt kínál a Java alapú alkalmazásokhoz. A Spring Boot \cite{jovanovic2017java} célja a Spring alkalmazás fejlesztés egyszerűsítése. Megtalálhatóak benne a következő tulajdonságok és különlegességek \cite{nodejsspring}:
\begin{itemize}
	\item Automatikus konfigurációk - Az alkalmazások Springként való működése érdekében.
	\item Indítófüggőségek - Biztosítja a felhasználóknak a szükséges függőségek (dependency) beillesztését. Ilyen lehet a Maven, Hibernate validátor, adatbázis elérések stb.
	\item Parancssori tolmács.
	\item Működtetés - A console-ba megjelennek az alkalmazás működésével kapcsolatos információk. Ilyen információ lehet a hiba, az elvégzett művelet stb.
	\item Radikálisan gyorsabb és széles körben hozzáférhető, érthető Spring fejlesztést nyújt. 
	\item Számos funkciót kínál: beágyazott szervereket, metrikákat, ellenőrzéseket, külső konfigurációkat.
	\item Egyszerű, minden eszköz és operációs rendszer támogatja.
	\item Beépített nyelvbiztonsági funkciókkal rendelkezik, amelyeket a Java Compiler beágyaz.
	\item Robusztus kódot alkalmaz.
	\item Integrációs képesség jó.
	\item Beágyazott HTTP-kiszolgálókat, például Jetty, Tomcat használ és egyszerűen teszteli a webes alkalmazásokat.
\end{itemize}

\subsection{Spring Boot vs NodeJs}

A Spring Boot és a NodeJs közötti néhány előnyt \cite{nodespringcomp} az \ref{tab:tableSpringNodeJsPros} táblázatban tudjuk megtekinteni.
\begin{table}[H]
	\begin{footnotesize}
		\begin{center}
			\caption[Előnyök Spring Boot és NodeJS között]{\textit{Előnyök Spring Boot és NodeJS között}}
			\label{tab:tableSpringNodeJsPros}
			\begin{tabular}{p{0.5cm}|p{6cm}|p{6cm}}
				\textbf & \textbf{Spring Boot} & \textbf{NodeJs}\\
				\hline
				1 & Java nyelv már nagyon ismeretes így ha elakadás van könnyű segítséget kérni/keresni& A JavaScript futásidejének köszönhetően gyors az adatfeldolgozásban\\
				\hline
				2 & Többszálú programozás & Memória takarékos\\
				\hline
				3 & Nagyszámú erőforrás áll rendelkezésre (például: az autentikáció már előre megvan írva) & NPM(Node Package Manager)-nek köszönhetően egyre több szolgáltatás érhető el. \\
			\end{tabular}
		\end{center}
	\end{footnotesize}
\end{table}

Minden keretrendszer között találhatunk előnyöket is meg hátrányokat is. A Spring Boot és a NodeJs között előnyöket az \ref{tab:tableSpringNodeJsPros} táblázatban már tárgyaltuk. A következőkbe néhány hátrányról \cite{nodespringcomp} lesz szó: 
\begin{itemize}
	\item A NodeJS nem tud hatékonyan teljesíteni nagy számítások esetén.
	\item A NodeJS mivel még új technika, ezért nincs teljesen kifelesztve, tesztelve így sok olyan hibába ütközhetünk amire nincs ideális megoldás.
	\item A Spring Boot legnagyobb hátránya, hogy a memóriaigénye nagyon nagy.
	\item A Spring Boot egy másik hátránya, hogy a hiba keresés meglehetősen nehéz mivel sok beépített kód található benne.
\end{itemize}

Összefoglalva mindkét technológia a maga módján megfelelő különböző alkalmazások tervezésénél. Ha az alkalmazás sok bemeneti/kimeneti feladatot (például: sok regisztráció) tartalmaz akkor mindenképp a NodeJs-t érdemes választani. Viszont ha biztonságos és önálló alkalmazást tervezünk amely intenzív CPU használatot igényel akkor a Spring Boottal érdemes foglalkozni.