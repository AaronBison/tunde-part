A dolgozat egy webalkalmazást mutat be, amely a Sapientia Erdélyi Magyar Tudományegyetem Marosvásárhely-i karának a 3D virtuális túráját foglalja magába. 

A mai világban az emberek napjait nagyban befolyásolja a digitalizáció. A legtöbb embernél található legalább egy okostelefon, számítógép, laptop amelyek mellé társul az internet is így a lehetőségek száma végtelen.

A digitalizáció az emberek számára nagyon sok jó dolgot vezet be. Rengeteg problémát tudunk megoldani az internet és a digitális eszközök segítségével, mint például: számlák fizetése, távoli rokonokkal könnyebb a kapcsolattartás és nem utolsó sorban a virtuális túrák segítségével eltudunk jutni olyan helyekre ahová nem biztos, hogy az életben lesz lehetőségünk.

A dolgozatban egy webes applikációról van szó amley felhasznál egy 3D modellt az egyetemről létrehozva az egyetem úgy nevezett virtuális túráját. A túrán bárki résztvehet ezáltal betekintést nyerhet az egyetem fala mögé. Ezen kívűl informatív jelleggel is rendelkezik, mivel az alkalmazás számos információt megjelenít az egyetemmel kapcsolatban, mint például: különböző események (Sapi-Line-Tracer), szakkoordinátorok nevei, elérhetőségei.

Az alkalmazás ugyanakkor rendelkezik a "Vigyél el!" funkcionalitással, amely a felhasználó által kiválasztott helységhez mutatja meg az odavezető utat, sok segítséget nyújtva az egyetemre látogatoknak.Ezen kívül arra is van lehetőség, hogy a felhasználók saját maguk lépegessenek az egyetem modelljén így még jobban körbe tudják járni azt.

A rendszer webes felelületre készült és ennek köszönhetően majdnem minden eszközön meglehet tekinteni, úgy a számítógépen mint a laptopon és nem utolsó sorban a telefonon is. Az alkalmazás sok segítséget nyújthat az újjonnan érkező egyetemistáknak, vendég diákoknak és a vendég tanároknak is az egyetem fő épületében való eligazodásnál. 

\textbf{Kulcsszavak}: webalkalmazás, 3D modell, virtuális túra.