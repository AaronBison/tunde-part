\section{Funkcionalitások}

Alkalmazásomon belül több fajta funkcionalitás jelenik meg. Első sorban nincs regisztrációhoz kötve a felhasználó. Ez azt akarja jelenteni, hogy mindenki előtt nyitva áll az alkalmazás. Viszont vannak különleges joggal rendelkező felhasználók, akik be tudnak jelentkezni és ennek következtében több funkcionalitást képesek használni.

Az alkalmazás három csoportba osztja a felhasználókat, vistor, user és admin. A három csoport közül kettőnek van leheteősége bejelentkezésre (admin, user). Az admin felhasználó elér minden funkcionalitást. Az a jellegzetessége, hogy új felhasználót csak ő tud hozzá adni és csak ő tud törölni is. A hozzáadás úgy történik, hogy megad minden adatot és egy gombra kattintva az új adatok bekerülnek az adatbázis rendszerbe. Közben az új felhasználó fog kapni egy email-t amellyel eltudja fogadni a jelentkeztetést. Ügyelni kell, hogy az email érvényessége időhöz kötött. A törlés csak egyszerűen kiválasztással történik. Egyetlen felhasználó sem tudja törölni saját magát még az admin joggal rendelkezők sem.

A következőkben a user joggal rendelkező felhasználók lehetőségeit részletezem. Ezeknek az embereknek lehetőségük van hozzáadni új 3D modelleket az egyetemről, eseményeket, órarend linkeket és nem utolsó sorban az oldalon található fejléc képének módosítására is képesek. A tanszékek, szakok leírásának megadásai is lehetővé válik számukra. Sőt ha netán új szak indul azt is hozzá lehet adni.

A bejelentkező felhasználók számára természetesen van kijelentkező opció is. Ezen funkciók mellett a fejlesztés során, valószínűleg sokkal több lesz. Egyelőre a cél az egyszerűségben rejlik. Tudni kell, hogy a modellnél, tanszékeknél, szakoknál és más fontos információknál elérhetőségeknél mindig az adatbázisban szereplő legújabb adat jelenik meg. 

A következőkben tárgyalom a visitor(látogató) felhasználó által elérhető funkciókat. Az első és legfontosabb, hogy megjelenik egy 3D modell az egyetem épületéről amelyen a felhasználók tudnak nézelődni. Tudják nagyítani, kicsinyíteni, sőt még lehetőségük lesz arra is, hogy az épületet körbe tudják járni. 

Mindenki számára megjelenik a bejelentkezési lehetőség is. Viszont ezt a funkciót csak azok használhatják akik rendelkeznek felhasználónévvel és jelszóval.

A második legfontosabb funkció a tanszékek, szakok megjelenítése. Ez egy külön oldalon lesz. A felhasználók itt láthatnak egy leírást és az elérhetőségeket egy adott tanszékről. Ezek mellett egy útvonalat is megtekinthetnek a 3D modellen belül így amikor oda érkeznek az egyetemre nem kell annyira keresgélni, hogy mi hol található. 

Mindenki számára elérhető lesz az esemény naptár is. Ezen részen jelennek meg azok az események amelyeket az egyetem szervez a diákok részére. Látható lesz az események pontos dátuma, helyszíne, ára és itt is megjelenik egy olyan opció is amely a 3D modellen megmutatja az útvonalat. Ez azért jó mert ha egy idegen ember érkezik egy ilyen eseményre, például versenyre, akkor könnyebben elfog tudni igazodni, hogy hová is kell mennie.

Utolsó funkcionalitásnak egy vélemény nyilvánítást tettem be. Itt minden felhasználó, név nélkül tudja közölni az alkalmazással kapcsolatos észrevételeit. Ha valaki véleményt szeretne írni akkor értékelnie is kell egy skálán, hogy szerinte mennyire hasznos az alkalmazás. A vélemények lehetnek pozitívak és negatívak is mindezek mellett új ötlet javaslatokat is lehet írni. Figyelembe véve a véleményeket lehetőség van jobbá fejleszteni az alkalmazást. A felhasználók nem csak véleményt tudnak nyilvánítani, hanem a mások által adottakat el is tudják olvasni. A vélemények szerkesztésére nem lesz lehetőség.

A funkcionalitásokból is látszik, hogy a céloknak megfelelően próbáltam megfelelni. Törekedtem az egyszerűségre átláthatóságra. Kiderült, hogy a regisztrációs rész nem egyedi módon történik meg. Az alkalmazás megpróbál minden olyan lehetőséget magába foglalni ami a diákok számára elérhető kell legyen.