	\paragraph{}
	A mai gyorsan fejlődő világunkban a digitális eszközök kezdik átvenni az uralmat. Nem sok olyan háztartás van ahol nincs egyáltalán legalább egy telefon, számítógép, laptop, okos tv stb. Az elmúlt években a digitális világ annyira fejlett lett, hogy lassan már ki sem kell mozdulnunk a házból és mindent eltudunk végezni. Például ki tudjuk fizetni a számláinkat, be tudunk vásárolni stb. Ezek mellett egyes múzeumokat, iskolákat, kastélyokat, látványosságokat is betudunk járni otthon a négy fal között a 3D virtuális túrák segítségével. De mit is jelent, mire is használják ezeket a 3D virtuális túra modelleket?
	\paragraph{}
	A 3D jelentése három dimenzió. Ez egy modell, amely tulajdonképpen matematikailag ábrázol egy háromdimenziós objektumot. Ez lehet épület, virág, ember stb. A 3D modelleket széles körben használnak az orvostudományban, magasabb szintű gyakorlati és elméleti kompetenciák elérésére. A 3D modellekkel nem csak tárgyakat hanem emberi folyamatokat is le lehet szimulálni. Az orvos tudományban nagyon szépen letudják előre játszani az operációkat így biztosabban fognak neki az adott műtétnek mivel jobban tudják kezelni a kockázatokat. Ilyen modellek felhasználásával egy adott problémát is jobban betudnak mutatni az emberek számára. Ilyen példa lehet egy hajóforgalom szimulációs rendszer amely figyelembe veszi a hajókat, vízfelületet és az időjárási viszonyokat. Ezzel a rendszerrel próbálták megmutatni az olajszennyezést a tengerekben, óceánokban.\cite{dedov2017design} Ezen modellek segítségével a világ látványosságai elérhetővé válhatnak azon emberek számára is akik nem jutnak el az eredeti országba, városba, hogy megtudják tekinteni az adott látványosságot.
	\paragraph{}
	A tour (virtuális túra) szó jelentése utazás, kirándulás. A mai világban túrázni nem csak a való életben lehet hanem a virtuális világban is. Egy ilyen virtuális túra célja, hogy fejlettebb szimulációs technikák segítségével a nézők életképes 3D-s képet kapjanak a meglévő helyről. Egy híres és megemlítendő példa a Second-Life ahol a felhasználók közötti interakció avatárokon keresztül zajlik. Ez azért megemlítendő példa mivel ezen alkalmazáson belül interaktív 3D tour-ok vannak leképezve.\cite{moloo20163d}
	\paragraph{}
	A 3D és a tour(virtuális túra) szavak összetételéből jön ki a 3D tour(3D virtuális túra) amely azt tükrözi, hogy a virtuális világban tudunk megnézni adott épületeket, kilátásokat, látványosságokat. Világunk fejlődése próbálja biztosítani, hogy egyetlen ember se maradjon le azon helyekről ahová nem tud eljutni. Habár ezen virtuális 3D modelleken alapuló világ még nem tökéletes de folyamatosan fejlődik. A dolgozatomon belül egy ilyen modellről lesz szó amely a Sapientia Erdélyi Magyar Tudományegyetem Marosvásárhely-i karának egy részét tartalmazza. A projekt két részre van bontva. Egy személy készíti el maga a modellt míg egy másik társ a felhasználói felületet. Ezen dolgozatban a felhasználói felületről lesz inkább szó. 
	\paragraph{}
	A projekt célja, hogy a felhasználók a saját házukból is betekintést tudjanak nyerni az egyetem falai mögé is. Ez elsősorban az új felvételizőknek és az első éves egyetemistáknak lenne hasznos, mivel ez által már otthon neki foghatnak áttekinteni az egyetem különböző részeit mint például, hogy hol található egy adott tanszék, vagy hol található a dékáni hivatal, vagy hogy hol található a pénztár. Úgy gondolom, hogy ez az alkalmazás hasznos tud lenni a diákok részére sőt a szakkoordinátorok dolgát is megtudja annyiban könnyíteni, hogy az újonnan érkező diákok nem kell mindig tőlük kérdezzék, hogy hol találnak meg egy adott részleget.
	\paragraph{}
	Több okot is fellehet sorolni annak érdekében, hogy miért is használatosak ezek a 3D modellekkel megalkotott virtuális túrák. Első sorban az új diákok már otthonról be tudnak nézni az egyetem falai mögé így amikor elérkeznek az új év kezdéséhez akkor otthonosabban érezhetik magukat. Ez azért történhet meg mert már fogják tudni, hogy mit hol találnak nem kell segítséget kérjenek. Egy másik ok, hogyha az adott egyetem rendelkezik egy ilyen típusú modellel akkor jobban felhívhatja a figyelmet a diákok számára. Ez által az egyetem népszerűsítése is megtörténik. Ismert, hogy egyes szülők nehezen engedik el gyermeküket egyetemre mert féltik. Viszont az, hogy látnak egy  képet az egyetemről megnyugtathatja őket így bátrabban biztathatják gyermeküket, hogy menjen el az adott egyetemre. 
	\paragraph{}
	Összefoglalva egy ilyen 3D modell amely az egyetemet ábrázolja és bemutatja hasznos lehet, az egyetem, a diákok és szüleik számára is. Egy ilyen alkalmazás esetében nem csak maga a modell jelenik meg hanem rajta kívül számos fontos információ, elérhetőség is. Dolgozatom célja bemutatni a Sapientia Erdélyi Magyar Tudományegyetem 3D Virtual Tour alkalmazás megalkotását, implementálását és nem utolsó sorban a felhasznált technológiákat
	
	