A mai gyorsan fejlődő világunkban a digitális eszközök a mindennapok elengedhetetlen részei. Nem sok olyan háztartás van ahol nincs egyáltalán legalább egy telefon, számítógép, laptop, okos tv. Az elmúlt években a digitális világ annyira fejlett lett, hogy lassan már ki sem kell mozdulnunk a házból és mindent eltudunk végezni. Például ki tudjuk fizetni a számláinkat, be tudunk vásárolni. Ezekből is következtethetünk, arra hogy egy jó internet kapcsolattal és egy közepes teljesítményű számítógép mellett már egy életet is letudunk élni.

Az elmúlt két évtizedben a „virtuális múzeum” \cite{kersten2017development} fogalmának meghatározása a gyors technológiai fejlődés következtében megváltozott. A mai rendelkezésre álló 3D technológiák a virtuális múzeumok esetén már nem csupán a gyűjtemények bemutatása az interneten vagy egy kiállítás virtuális bemutatója panorámás fotózás segítségével. Egy virtuális múzeum nem csak az oda látogatóknak hasznos, hanem már tanórákon is feltudják használni a tanárok mivel nem mindig sikerül elvinni a diákokat egy adott városba így egy ilyen virtuális túra segítségével a tanár betudja mutatni az adott múzeum látványosságait. Sőt a gyerekek ezáltal otthon is eltudnak barangolni más országok, kontinensek múzeumaiba virtuálisan.

Világunk fejlődése próbálja biztosítani, hogy egyetlen ember se maradjon le azon helyekről ahová nem tud eljutni, így egyes múzeumokat, iskolákat, kastélyokat, látványosságokat is betudunk járni otthon a négy fal között a 3D virtuális túrák segítségével.

Több okot is fel lehet sorolni annak érdekében, hogy miért is használatosak ezek a 3D modellekkel megalkotott virtuális túrák. Első sorban az új diákok már otthonról be tudnak nézni az egyetem falai mögé így amikor elérkeznek az új év kezdéséhez akkor otthonosabban érezhetik magukat. Ez azért történhet meg mert már fogják tudni, hogy mit hol találnak nem kell segítséget kérjenek. Egy másik ok, hogyha az adott egyetem rendelkezik egy ilyen típusú modellel akkor jobban felhívhatja a figyelmet a diákok számára. Ez által az egyetem népszerűsítése is megtörténik. Ez mellet a vendég kutató, tanár vagy akár hallgató is könnyeben eltud igazodni az egyetem főépületében.

A 3D jelentése három dimenzió. Ez egy modell, amely tulajdonképpen matematikailag ábrázol egy háromdimenziós objektumot, amely lehet épület, virág, vagy akár egy ember is. A 3D modelleket széles körben használják az orvostudományban, magasabb szintű gyakorlati és elméleti kompetenciák elérésében. A 3D modellekkel nem csak tárgyakat hanem emberi folyamatokat is le lehet szimulálni.

Sok esetben egy szimulált 3D modell segít egy adott problémát jobban átlátni. Ilyen példa lehet amikor az orvos tudományban \cite{dedov2017design} nagyon szépen letudják előre játszani a műtéteket így biztosabbak a dolgukba és a kockázati lehetőségeket is csökkenthetik. Egy másik példa lehet egy hajóforgalom szimulációs rendszer \cite{dedov2017design}, amely figyelembe veszi a hajókat, vízfelületet és az időjárási viszonyokat. Ezzel a rendszerrel próbálták megmutatni az olajszennyezést \cite{dedov2017design} a tengerekben, óceánokban. Ezen modellek segítségével a világ látványosságai elérhetővé válhatnak azon emberek számára is, akik nem jutnak el az eredeti országba, városba, hogy megtudják tekinteni az adott látványosságot.

A számítógépes grafikát \cite{moloo20163d} sok terülten alkalmazzák mint például az interaktív médiatervezésben, 3D túrák készítésében és videojáték iparban. Az elmúlt néhány év során számos technológia jelent meg a 3D virtuális túrák megjelenítésére az interneten. A virtuális túrák a felhasználóknak biztosítják az életszerű 3D környezeteket.

A tour szó jelentése utazás, kirándulás. A mai világban túrázni a virtuális világban is lehet. Egy ilyen virtuális túra célja, hogy fejlettebb szimulációs technikák segítségével a nézők élethű 3D-s képet kapjanak a meglévő helyről. Egy híres példa a Second-Life \cite{moloo20163d} ahol a felhasználók közötti interakció avatárokon keresztül zajlik. Ezen alkalmazáson belül interaktív 3D tour-ok vannak leképezve.

A 3D és a tour (virtuális túra) szavak összetételéből jön ki a 3D tour (3D virtuális túra) amely azt tükrözi, hogy a virtuális világban tudunk megnézni adott épületeket, kilátásokat, látványosságokat. Habár ezen virtuális 3D modelleken alapuló világ még nem tökéletes de folyamatosan fejlődik. A dolgozatomon belül egy ilyen modellről lesz szó amely a Sapientia Erdélyi Magyar Tudományegyetem Marosvásárhely-i karának egy részét tartalmazza. A projekt két részre van bontva, amely két államvizsga dolgozatot eredményez a 3D modell és a felhasználói felület. Közös munka a modellel történő attrakciók megoldása. Ilyen attrakció lehet a modellben való mozgás, közlekedés. Ezen dolgozatban a felhasználói felületről lesz szó. 

Egy ilyen alkalmazás esetében nem csak maga a modell jelenik meg, hanem rajta kívül számos fontos információ, elérhetőség is. Dolgozatom célja bemutatni a Sapientia Erdélyi Magyar Tudományegyetem 3D Virtual Tour alkalmazás megalkotását, implementálását és nem utolsó sorban a felhasznált technológiákat.

