\paragraph{}
A világhalón való keresés során sok tanulmányt találtam, amely leírja hogy egy virtuális túra egyetemeken, múzeumokban, különböző látványosságoknál mekkora befolyásoló képességgel rendelkezik. Megtudhattam, hogy sok egyetemnek van ilyen jellegű túrája más más típusban. Van amelyik egyetem a virtuális túrát képek sorozatában képzelte el, van amelyik panoráma képekben, van amelyik videókban és nem utolsó sorban jönnek azok akik 3D modellekkel valósították meg az egyetemük virtuális túráját. A következőkben pár ilyen jellegű kutatásról írnék pár szót.

\section{Web alapú campus a Telkom Egyetem bemutatásához}
\paragraph{}
Indonézia egyik legnagyobb magánegyeteme (Telkom University (Tel-U)) is a virtuális 3D túrát alkalmazza az új diákok oda vonzására. A túrák tartalmaznak videó és képsorozatokat plusz 3d alapú modelleket is. Tulajdonképpen egy 3D web alapú túrát dolgoztak ki a 3D Vista használatával. 
\paragraph{}
Miközben fejlesztették, ezt a túrát kutatásokat végeztek, hogy mivel lenne jó elkészíteni, más egyetemek milyen technológiákat alkalmaztak hazai területeken. A következő eredmények születtek: 
\begin{itemize}
	\item RF Rahmat: épületekről információ szolgáltatásokat kiviteleztek Universitas Sumatera Utara (USU) környezetben.
	\item Moloo: virtuális túrát hoz létre a WebGL és a SketchUp segítségével.
	\item Fujita: Mobile robotokat használ virtuális túrák elkészítéséhez.
	
\end{itemize}
\paragraph{}
Szeliski R aki a fotó technikákat mutatta be, szerint is a nagy felbontású fotók és a 3D modellek együttes használatával nagyobb érdeklődési kört lehet elérni, mint ha csak külön használnák őket. 
Az egyetem kiemelt kutatási stratégia terve volt, hogy ne csak a diákokat vonzzák magukhoz, hanem az IKT (Információs és kommunikációs technológia) technológiák fejlesztésében is érjenek el fejlődéseket. Ennek érdekében használtak 3D modelleket és nagy felbontású fotókat együttesen. A diákok számára nagy érdekességnek számított és ezzel elérték azt, hogy a diákok érdeklődését felkeltették.\cite{perdana2019implementation}

\section{3D virtuális túra a Mauritiusi Egyetemen a WebGL használatával}
\paragraph{}
A számítógépes grafikát sok terülten alkalmazzák mint például a interaktív médiatervezésben, 3D túrák készítésében és videojáték iparban. Az elmúlt néhány év során számos technológia jelent meg a 3D virtuális túrák megjelenítésére az interneten. A virtuális túrák a felhasználóknak biztosítják az életszerű 3D környezeteket.
\paragraph{} 
Mauritius-i egyetem is megalkotta saját 3D virtuális túráját. A megalkotást WebGL segítségével történt. A WebGL (Web Graphics Library) könyvtár 3D-s valós idejű megjelenítést kínál. A WebGL az OpenGL-ből származik , és API-t biztosít a 3D grafikához.  A WebGL nem teljesen stabil  és néhány algoritmusa nem hatékonyan látja el a feladatait. Különböző böngészők memóriahasználata és és végrehajtási ideje eltérő amelyek mellékhatásokat idézhetnek elő a megírt alkalmazásban.
\paragraph{}
Az egyetem is szintén valós időbeni megjelenítést alkalmaz és a megfelelő működés érdekében tesztelték az alkalmazást teljesítmény szempontjából, amikor több felhasználó csatlakozik egyidejűleg.  A teszt eredményiből az egyetem arra következtetett, hogy minél összetettebb az objektum és ha még textúrával is rendelkezik akkor a teljesítmény nagyon csökken mivel a modell így nagyon nagy erőforrást igényel. 
\paragraph{}
Az egyetem a megvalósításnak az egyik legegyszerűbb formáját választotta amellyel sikeresen elérték céljukat. Megalkották a virtuális túrájukat és ezt felhasználva felkeltették az jövendőbeli egyetemisták érdeklődését.\cite{moloo20163d}

\section{A játéktechnika alkalmazása virtuális túrák esetén}
\paragraph{}
A virtuális túrák információt nyújtanak multimédiás úton a felhasználóknak, azt a benyomást keltve hogy valós időben navigálnak különböző helyeken. Egy sikeres túra jelentése, hogy a felhasználó eltudja hinni, hogy ő habár virtuálisan is de járt abban az adott helyiségben. Ahhoz hogy ez a benyomás létre jőjön a modell pontos ábrázolást kell tartalmazzon az adott helyiségről. Ezen tanulmány szerint is az ilyen túrák felhasználhatóak létesítmények népszerűsítésére mivel egy interaktív élményt biztosítanak. Az ilyen jellegű túrákat össze lehet hasonlítani a számítógépes játékokkal is. Hiszen ezek a játékok annyiban különböznek, hogy nem valós időben és nem valós helyeken történnek. Persze vannak kivételek, ahol a játék egy egy része magába foglal egy valós helyszínt is. 
\paragraph{}
Az Egyesült Királyság számos egyeteme használ ilyen jellegű túrákat annak érdekében, hogy az emberek távolról is betekintést tudjanak nyerni az egyetemek világába. A kutatás szerint tizennégy britt egyetemet tekintve mindegyik weboldalán találtak állóképgalérát és videogalériát viszont meglepő számnak számított hogy 3600 interaktív túrát is találtak. Azért meglepő szám mivel ezek a túrák még nem igazán elterjedtek.
\paragraph{}
Ezen tanulmány felmérte az Egyesült Királyság egyetemei körében mennyire használatosak a virtuális túrák. Kiderült, hogy a virtuális egyetemi túrák interaktívabbá tették az egyetemek weboldalait és nagyobb fokú szolgáltatást biztosítottak a felhasználók számára hiszen nem csak képeket, videókat láttak az egyetemekről, hanem maga az egyetemet is. Az egyetemek ezen túrák megalkotásában teljes mértékben kihasználták  a számítógépes játékokhoz használt grafikai eszközöket. Az modellek segítenek bemutatni az egyetemeket, plusz útvonalakat is tervezhetnek a modellben így már előre fogják tudni, hogy mit hol találnak a felhasználók. A jövőben valószínűleg majdnem minden egyetem fogja alkalmazni az ilyen jellegű túrákat.\cite{maines2015application}
