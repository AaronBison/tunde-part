\usepackage{graphicx}
\graphicspath{ {figures/} }
\usepackage[left=2.5cm,right=2cm,top=2.5cm]{geometry}

\usepackage[utf8]{inputenc}
\usepackage{t1enc}
\usepackage[magyar]{babel}
\linespread{1.5}
\usepackage{url}

\usepackage{footnote}
\usepackage{subfigure}
\usepackage{float}
\usepackage[]{algorithm2e}
\usepackage{amsmath}
\usepackage{mathptmx}

\usepackage{pdfpages}

\usepackage{xcolor}
\usepackage{hyperref}
\hypersetup{
    colorlinks,
    linkcolor=black,
    citecolor=black,
	urlcolor={blue!80!black},
	unicode=true
}
% 
\urlstyle{same}

\usepackage{listings}
\definecolor{dkgreen}{rgb}{0,0.6,0}
\definecolor{gray}{rgb}{0.5,0.5,0.5}
\definecolor{mauve}{rgb}{0.58,0,0.82}
\definecolor{light-gray}{gray}{0.25}

\lstdefinestyle{yaml}{	
	backgroundcolor=\color{white}, % choose the background color;
	basicstyle=\fontsize{8}{8}\ttfamily,% the size of the fonts that are used for the code
	breakatwhitespace=false, % sets if automatic breaks should only happen at whitespace
	breaklines=true, % sets automatic line breaking
	commentstyle=\color{dkgreen},  % comment style
	deletekeywords={...},  % if you want to delete keywords from the given language
	escapeinside={\%}{)},  % if you want to add LaTeX within your code
	extendedchars=true,  % lets you use non-ASCII characters; for 8-bits encodings only, does not work with UTF-8
	frame=none,	 	 % adds a frame around the code
	keepspaces=true, % keeps spaces in text, useful for keeping indentation of code (possibly needs columns=flexible)
	keywordstyle=\color{blue}\bfseries, % keyword style
	otherkeywords={*,...}, % if you want to add more keywords to the set
	numbers=none,  % where to put the line-numbers; possible values are (none, left, right)
	numbersep=5pt, % how far the line-numbers are from the code
	numberstyle=\tiny\color{gray}, % the style that is used for the line-numbers
	rulecolor=\color{black}, % if not set, the frame-color may be changed on line-breaks within not-black text (e.g. comments (green here))
	showspaces=false,% show spaces everywhere adding particular underscores; it overrides 'showstringspaces'
	showstringspaces=false,  % underline spaces within strings only
	showtabs=false,  % show tabs within strings adding particular underscores
	%stepnumber=1,  % the step between two line-numbers. If it's 1, each line will be numbered
	stringstyle=\color{mauve}, % string literal style
	tabsize=2,
	columns=fullflexible  % Using fixed column width (for e.g. nice alignment)
	sensitive = true,
	morekeywords={name, runs-on, on, jobs, build, run, steps, uses}
}
\lstset{}